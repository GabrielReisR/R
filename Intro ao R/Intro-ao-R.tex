% Options for packages loaded elsewhere
\PassOptionsToPackage{unicode}{hyperref}
\PassOptionsToPackage{hyphens}{url}
%
\documentclass[
]{article}
\usepackage{lmodern}
\usepackage{amssymb,amsmath}
\usepackage{ifxetex,ifluatex}
\ifnum 0\ifxetex 1\fi\ifluatex 1\fi=0 % if pdftex
  \usepackage[T1]{fontenc}
  \usepackage[utf8]{inputenc}
  \usepackage{textcomp} % provide euro and other symbols
\else % if luatex or xetex
  \usepackage{unicode-math}
  \defaultfontfeatures{Scale=MatchLowercase}
  \defaultfontfeatures[\rmfamily]{Ligatures=TeX,Scale=1}
\fi
% Use upquote if available, for straight quotes in verbatim environments
\IfFileExists{upquote.sty}{\usepackage{upquote}}{}
\IfFileExists{microtype.sty}{% use microtype if available
  \usepackage[]{microtype}
  \UseMicrotypeSet[protrusion]{basicmath} % disable protrusion for tt fonts
}{}
\makeatletter
\@ifundefined{KOMAClassName}{% if non-KOMA class
  \IfFileExists{parskip.sty}{%
    \usepackage{parskip}
  }{% else
    \setlength{\parindent}{0pt}
    \setlength{\parskip}{6pt plus 2pt minus 1pt}}
}{% if KOMA class
  \KOMAoptions{parskip=half}}
\makeatother
\usepackage{xcolor}
\IfFileExists{xurl.sty}{\usepackage{xurl}}{} % add URL line breaks if available
\IfFileExists{bookmark.sty}{\usepackage{bookmark}}{\usepackage{hyperref}}
\hypersetup{
  pdftitle={Introdução ao R},
  pdfauthor={Gabriel R. R. \& Wagner L. M.},
  hidelinks,
  pdfcreator={LaTeX via pandoc}}
\urlstyle{same} % disable monospaced font for URLs
\usepackage[margin=1in]{geometry}
\usepackage{color}
\usepackage{fancyvrb}
\newcommand{\VerbBar}{|}
\newcommand{\VERB}{\Verb[commandchars=\\\{\}]}
\DefineVerbatimEnvironment{Highlighting}{Verbatim}{commandchars=\\\{\}}
% Add ',fontsize=\small' for more characters per line
\usepackage{framed}
\definecolor{shadecolor}{RGB}{248,248,248}
\newenvironment{Shaded}{\begin{snugshade}}{\end{snugshade}}
\newcommand{\AlertTok}[1]{\textcolor[rgb]{0.94,0.16,0.16}{#1}}
\newcommand{\AnnotationTok}[1]{\textcolor[rgb]{0.56,0.35,0.01}{\textbf{\textit{#1}}}}
\newcommand{\AttributeTok}[1]{\textcolor[rgb]{0.77,0.63,0.00}{#1}}
\newcommand{\BaseNTok}[1]{\textcolor[rgb]{0.00,0.00,0.81}{#1}}
\newcommand{\BuiltInTok}[1]{#1}
\newcommand{\CharTok}[1]{\textcolor[rgb]{0.31,0.60,0.02}{#1}}
\newcommand{\CommentTok}[1]{\textcolor[rgb]{0.56,0.35,0.01}{\textit{#1}}}
\newcommand{\CommentVarTok}[1]{\textcolor[rgb]{0.56,0.35,0.01}{\textbf{\textit{#1}}}}
\newcommand{\ConstantTok}[1]{\textcolor[rgb]{0.00,0.00,0.00}{#1}}
\newcommand{\ControlFlowTok}[1]{\textcolor[rgb]{0.13,0.29,0.53}{\textbf{#1}}}
\newcommand{\DataTypeTok}[1]{\textcolor[rgb]{0.13,0.29,0.53}{#1}}
\newcommand{\DecValTok}[1]{\textcolor[rgb]{0.00,0.00,0.81}{#1}}
\newcommand{\DocumentationTok}[1]{\textcolor[rgb]{0.56,0.35,0.01}{\textbf{\textit{#1}}}}
\newcommand{\ErrorTok}[1]{\textcolor[rgb]{0.64,0.00,0.00}{\textbf{#1}}}
\newcommand{\ExtensionTok}[1]{#1}
\newcommand{\FloatTok}[1]{\textcolor[rgb]{0.00,0.00,0.81}{#1}}
\newcommand{\FunctionTok}[1]{\textcolor[rgb]{0.00,0.00,0.00}{#1}}
\newcommand{\ImportTok}[1]{#1}
\newcommand{\InformationTok}[1]{\textcolor[rgb]{0.56,0.35,0.01}{\textbf{\textit{#1}}}}
\newcommand{\KeywordTok}[1]{\textcolor[rgb]{0.13,0.29,0.53}{\textbf{#1}}}
\newcommand{\NormalTok}[1]{#1}
\newcommand{\OperatorTok}[1]{\textcolor[rgb]{0.81,0.36,0.00}{\textbf{#1}}}
\newcommand{\OtherTok}[1]{\textcolor[rgb]{0.56,0.35,0.01}{#1}}
\newcommand{\PreprocessorTok}[1]{\textcolor[rgb]{0.56,0.35,0.01}{\textit{#1}}}
\newcommand{\RegionMarkerTok}[1]{#1}
\newcommand{\SpecialCharTok}[1]{\textcolor[rgb]{0.00,0.00,0.00}{#1}}
\newcommand{\SpecialStringTok}[1]{\textcolor[rgb]{0.31,0.60,0.02}{#1}}
\newcommand{\StringTok}[1]{\textcolor[rgb]{0.31,0.60,0.02}{#1}}
\newcommand{\VariableTok}[1]{\textcolor[rgb]{0.00,0.00,0.00}{#1}}
\newcommand{\VerbatimStringTok}[1]{\textcolor[rgb]{0.31,0.60,0.02}{#1}}
\newcommand{\WarningTok}[1]{\textcolor[rgb]{0.56,0.35,0.01}{\textbf{\textit{#1}}}}
\usepackage{graphicx,grffile}
\makeatletter
\def\maxwidth{\ifdim\Gin@nat@width>\linewidth\linewidth\else\Gin@nat@width\fi}
\def\maxheight{\ifdim\Gin@nat@height>\textheight\textheight\else\Gin@nat@height\fi}
\makeatother
% Scale images if necessary, so that they will not overflow the page
% margins by default, and it is still possible to overwrite the defaults
% using explicit options in \includegraphics[width, height, ...]{}
\setkeys{Gin}{width=\maxwidth,height=\maxheight,keepaspectratio}
% Set default figure placement to htbp
\makeatletter
\def\fps@figure{htbp}
\makeatother
\setlength{\emergencystretch}{3em} % prevent overfull lines
\providecommand{\tightlist}{%
  \setlength{\itemsep}{0pt}\setlength{\parskip}{0pt}}
\setcounter{secnumdepth}{-\maxdimen} % remove section numbering

\title{Introdução ao R}
\author{Gabriel R. R. \& Wagner L. M.}
\date{6/27/2020}

\begin{document}
\maketitle

\hypertarget{considerauxe7uxf5es-iniciais}{%
\subsection{Considerações iniciais}\label{considerauxe7uxf5es-iniciais}}

O código inicial, em R, está disponível nesse link:
\url{https://github.com/wagnerLM/quanti2/blob/master/script\%20R\%20intro2.R}

Inspiradao em Torfs \& Brauer (2012) e Epskamp (2013) ``Introductions to
R''.

\hypertarget{iniciando-no-r}{%
\subsection{1. Iniciando no R}\label{iniciando-no-r}}

\hypertarget{o-que-uxe9-o-r}{%
\subsubsection{1.1 O que é o R?}\label{o-que-uxe9-o-r}}

O R é uma linguagem de programação altamente utilizada para estatística.
Ele permite realizar análises estatísticas, visualização de dados,
extração (mineração) de dados, e possui ainda outras muitas
funcionalidades. Além disso, o R é um software livre e gratuito que
conta com uma comunidade ativa de usuários ao redor do mundo.

Por essas características acima, novas atualizações e possibilidades de
análise são frequentes no R - o tornando uma ferramenta bem importante
para pesquisa científica, \emph{data science} etc.

Antes do R, o S havia sido criado em 1976 por John Chambers nos
Laboratórios Bell - laboratórios industriais americanos de propriedade
da Nokia. Em 1992 na Nova Zelândia, Ross Ihaka e Robert Gentleman
começaram a produzir o R \textbf{baseando-se fortemente no software S}.
Em 1995 a primeira versão de R havia sido lançada. A primeira versão
estável saiu em 2000.

A partir de 2018, Robert Chambers, criador do S, juntou-se ao time de
desenvolvimento do R. Qualquer informação sobre o R pode ser consultada
no site oficial do R: \url{https://www.r-project.org/}

Dentre as razões para utilizar o R estão:

\begin{itemize}
\tightlist
\item
  Capacidades gráficas muito sofisticadas e melhores que muitos
  softwares.
\item
  Possibilidade para desenvolver novas ferramentas.
\item
  Comunidade de usuários muito ativa e participativa.
\end{itemize}

\hypertarget{baixando-o-r}{%
\subsubsection{1.2 Baixando o R}\label{baixando-o-r}}

O R é gratuito. Para realizar o download do software, basta acessar o
site \url{https://cloud.r-project.org/} e escolher a versão apropriada
para sua máquina.

Para acessar o R, vamos utilizar o RStudio - uma interface que facilita
o uso do R. Da mesma forma, o RStudio é gratuito e pode ser baixado
aqui: \url{https://www.rstudio.com/products/rstudio/download/}

Para mexermos no R, basta abrir o RStudio.

\hypertarget{reconhecendo-os-quadrantes-do-rstudio}{%
\subsubsection{1.3 Reconhecendo os quadrantes do
RStudio}\label{reconhecendo-os-quadrantes-do-rstudio}}

Ao abrir o RStudio, você notará algumas divisões na tela. Cada divisão é
representa um \emph{quadrante} do R. Cada quadrante nos dá informações
diferentes.

\begin{figure}
\centering
\includegraphics{C:/Users/Marco2/Desktop/Gabriel/Data Science/Projetos/Grupo R/imagens/1.jpg}
\caption{Quadrantes do RStudio}
\end{figure}

\begin{itemize}
\tightlist
\item
  1 - Esquerdo-alto: editor ou script, você pode armazenar seus comandos
  e salvar os seus projetos.
\item
  2 - Direito-alto: espaço de trabalho e repositório de seus
  ``objetos''.
\item
  3 - Direito-baixo: arquivos de ajuda, imagens e gráficos, e
  informações dos pacotes.
\item
  4 - Esquerdo-baixo: console e comando, mostra os registros de
  atividades, saídas de análises e é onde o ``R'' funciona.
\end{itemize}

Vamos começar?

\hypertarget{abrindo-uma-nova-aba}{%
\subsubsection{1.4 Abrindo uma nova aba}\label{abrindo-uma-nova-aba}}

Caso queira iniciar outro trabalho, você pode abrir uma nova aba de
script no RStudio.

Para isso, basta ir acessar a aba do programa e clicar em
\texttt{File\ -\textgreater{}\ New\ File\ -\textgreater{}\ R\ Script}.
Você agora terá mais um script para utilizar.

\hypertarget{primeiros-passos-com-o-r}{%
\subsection{2. Primeiros passos com o
R}\label{primeiros-passos-com-o-r}}

\hypertarget{usando-o-r-como-uma-calculadora}{%
\subsubsection{2.1 Usando o R como uma
calculadora}\label{usando-o-r-como-uma-calculadora}}

Para começar, digite no campo 1 - esquerdo-alto (a partir de agora,
chamado de \emph{editor} ou \emph{script}) a seguinte operação:
\texttt{1\ +\ 1}

Agora, para o R reconhecer que você quer fazer o resultado dessa
operação, você precisa \emph{rodar esse código}.

\textbf{Rodando um código}: para seus comandos serem executados,
selecione ou o início da linha de código, ou o final da linha de código,
ou toda a linha de código, e então clique em \emph{Run}.

\begin{figure}
\centering
\includegraphics{C:/Users/Marco2/Desktop/Gabriel/Data Science/Projetos/Grupo R/imagens/2.jpg}
\caption{Rodando um comando}
\end{figure}

\begin{itemize}
\tightlist
\item
  Windows: O comando \texttt{Run} pode ser substituído por ``Ctrl +
  Enter''
\item
  Mac: O comando \texttt{Run} pode ser substituído por ``Cmd + Return''
\end{itemize}

Ficará assim:

\begin{Shaded}
\begin{Highlighting}[]
\DecValTok{1} \OperatorTok{+}\StringTok{ }\DecValTok{1}
\end{Highlighting}
\end{Shaded}

\begin{verbatim}
## [1] 2
\end{verbatim}

O resultado aparecerá no quadrante 4 - esquerdo-baixo (a partir de
agora, chamado de \textbf{console}): \texttt{{[}1{]}\ 2} Ou seja, 1 + 1
= 2.

É simples assim para utilizar o R como uma calculadora!

\emph{Todos os resultados aparecem no console. Não é necessário escrever
os comandos no editor de código (quadrante 1) - se preferir, pode
escrevê-los direto no console (quadrante 4).}

Para \textbf{somar}, usa-se \texttt{+}, como já mostrado.

\textbf{Subtrair} usando \texttt{-}. Por exemplo, \texttt{50\ -\ 25}:

\begin{Shaded}
\begin{Highlighting}[]
\DecValTok{50} \OperatorTok{-}\StringTok{ }\DecValTok{25}
\end{Highlighting}
\end{Shaded}

\begin{verbatim}
## [1] 25
\end{verbatim}

\textbf{Multiplicar} usando \texttt{*}. Por exemplo, \texttt{5\ *\ 5}

\begin{Shaded}
\begin{Highlighting}[]
\DecValTok{5} \OperatorTok{*}\StringTok{ }\DecValTok{5}
\end{Highlighting}
\end{Shaded}

\begin{verbatim}
## [1] 25
\end{verbatim}

\textbf{Dividir} usando \texttt{/}. Por exemplo, \texttt{50\ /\ 2}

\begin{Shaded}
\begin{Highlighting}[]
\DecValTok{50} \OperatorTok{/}\StringTok{ }\DecValTok{2}
\end{Highlighting}
\end{Shaded}

\begin{verbatim}
## [1] 25
\end{verbatim}

\textbf{Potencializar} usando \texttt{**} ou \texttt{\^{}x}. Por
exemplo, \texttt{5\ **\ 2} ou \texttt{5\^{}2}

\begin{Shaded}
\begin{Highlighting}[]
\DecValTok{5} \OperatorTok{**}\StringTok{ }\DecValTok{2}
\end{Highlighting}
\end{Shaded}

\begin{verbatim}
## [1] 25
\end{verbatim}

Assim, uma série de operações podem ser feitas.

Em cálculos maiores, os parênteses \texttt{(\ )} são utilizados para
priorizar os elementos que precisam ser calculados antes.

\begin{Shaded}
\begin{Highlighting}[]
\NormalTok{((}\DecValTok{10} \OperatorTok{*}\StringTok{ }\DecValTok{20}\NormalTok{) }\OperatorTok{/}\StringTok{ }\DecValTok{8}\NormalTok{) }\OperatorTok{^}\StringTok{ }\DecValTok{2}
\end{Highlighting}
\end{Shaded}

\begin{verbatim}
## [1] 625
\end{verbatim}

\hypertarget{objetos-no-r}{%
\subsection{3. Objetos no R}\label{objetos-no-r}}

Até agora trabalhamos apenas com \emph{números inteiros} e com cálculos.
É importante citar que o R é esperto suficiente para conseguir trabalhar
com \emph{texto} também.

Antes de aprender sobre isso, vamos aprender a criar um \emph{objeto}.

\hypertarget{criando-objetos}{%
\subsubsection{3.1 Criando objetos}\label{criando-objetos}}

Um objeto é uma coisa que se pode mudar, mexer e brincar com. Ele possui
atributos específicos, dependendo do que a gente coloca nele.

Para criar um objeto, primeiro se escolhe o seu nome e depois se
atribuem os seus valores.

\textbf{Ah, todas as linhas iniciadas em}\texttt{\#}\textbf{são
comentários, e não são processadas como comandos pelo software.}

\begin{Shaded}
\begin{Highlighting}[]
\CommentTok{# Esse é um comentário. Note que ele inicia com "#". O R não se importa com ele.}

\CommentTok{# Vamos agora criar um objeto chamado "objeto".}

\NormalTok{objeto <-}\StringTok{ }\DecValTok{10}
\end{Highlighting}
\end{Shaded}

Pronto, agora \texttt{objeto} possui valor \texttt{10}.

\begin{itemize}
\tightlist
\item
  Para criar objetos, é preciso definir seu \emph{nome} \texttt{objeto}
  ou \texttt{x} ou \texttt{nomeSuperLegal} (o R não se importa\ldots).
\item
  Depois, é necessário colocar o \emph{sinal de atribuição}
  \texttt{\textless{}-}, que pode significar \emph{possui o valor de}.
\item
  Finalmente, dá-se o \emph{valor do objeto}, que nesse caso é
  \texttt{10}.
\end{itemize}

\texttt{objeto\ \textless{}-\ 10} pode ser lido como \emph{criar}
\texttt{objeto} \emph{que} \textbf{possui o valor de}
(\texttt{\textless{}-}) \texttt{10}.

Depois de criado, se pedirmos apenas \texttt{objeto} para o R, olhe o
que ele nos retorna:

\begin{Shaded}
\begin{Highlighting}[]
\CommentTok{# Agora vamos apenas pedir objeto}

\NormalTok{objeto}
\end{Highlighting}
\end{Shaded}

\begin{verbatim}
## [1] 10
\end{verbatim}

O R nos retornou o valor 10! Ele entendeu que \texttt{objeto} agora é
\texttt{10}.

Para saber qual o tipo desse objeto, \emph{o que ele é}, é só pedir

\begin{Shaded}
\begin{Highlighting}[]
\CommentTok{# Agora vamos apenas para o R informar o que "objeto" é}

\CommentTok{# Observação: class() é uma função que retorna o tipo de objeto colocado nela.}

\KeywordTok{class}\NormalTok{(objeto)}
\end{Highlighting}
\end{Shaded}

\begin{verbatim}
## [1] "numeric"
\end{verbatim}

Que legal! O R nos disse que \texttt{objeto} é um numeral.

Como um numeral, podemos fazer operações com ele, certo?

\begin{Shaded}
\begin{Highlighting}[]
\NormalTok{objeto }\OperatorTok{+}\StringTok{ }\DecValTok{10}  \CommentTok{# Retorna 20}
\end{Highlighting}
\end{Shaded}

\begin{verbatim}
## [1] 20
\end{verbatim}

\begin{Shaded}
\begin{Highlighting}[]
\NormalTok{objeto }\OperatorTok{-}\StringTok{ }\DecValTok{10}  \CommentTok{# Retorna 0}
\end{Highlighting}
\end{Shaded}

\begin{verbatim}
## [1] 0
\end{verbatim}

\begin{Shaded}
\begin{Highlighting}[]
\NormalTok{objeto }\OperatorTok{/}\StringTok{ }\DecValTok{2}   \CommentTok{# Retorna 5}
\end{Highlighting}
\end{Shaded}

\begin{verbatim}
## [1] 5
\end{verbatim}

\begin{Shaded}
\begin{Highlighting}[]
\NormalTok{objeto }\OperatorTok{*}\StringTok{ }\DecValTok{5}   \CommentTok{# Retorna 50}
\end{Highlighting}
\end{Shaded}

\begin{verbatim}
## [1] 50
\end{verbatim}

\begin{Shaded}
\begin{Highlighting}[]
\NormalTok{objeto }\OperatorTok{/}\StringTok{ }\NormalTok{objeto }\CommentTok{# Retorna 1 (x / x = 1)}
\end{Highlighting}
\end{Shaded}

\begin{verbatim}
## [1] 1
\end{verbatim}

\hypertarget{objetos-interagindo-com-objetos}{%
\subsubsection{3.2 Objetos interagindo com
objetos}\label{objetos-interagindo-com-objetos}}

Vamos criar dois objetos, \texttt{a}com valor \texttt{50} e \texttt{b}
com valor \texttt{3} e pedir para eles interagirem.

Vimos que para criar um objeto, é só usar a fórmula
\texttt{objeto\ \textless{}-\ valor} (lê-se \emph{objeto possui o valor
de `valor'}).

\begin{Shaded}
\begin{Highlighting}[]
\NormalTok{a <-}\StringTok{ }\DecValTok{50}

\NormalTok{b <-}\StringTok{ }\DecValTok{3}

\CommentTok{# Podemos interagir com os objetos}

\NormalTok{a }\OperatorTok{/}\StringTok{ }\NormalTok{b   }\CommentTok{# 'a' dividido por 'b'}
\end{Highlighting}
\end{Shaded}

\begin{verbatim}
## [1] 16.66667
\end{verbatim}

\begin{Shaded}
\begin{Highlighting}[]
\NormalTok{a }\OperatorTok{**}\StringTok{ }\NormalTok{b  }\CommentTok{# 'a' elevado a 'b'}
\end{Highlighting}
\end{Shaded}

\begin{verbatim}
## [1] 125000
\end{verbatim}

\textbf{Exercícios para você fazer}

\begin{enumerate}
\def\labelenumi{\arabic{enumi}.}
\item
  Realize \texttt{a\ +\ b}.
\item
  Agora crie um novo objeto, \texttt{c}, que possua o valor de
  \texttt{a\ /\ b}.
\item
  Qual a classe desse novo objeto \texttt{c}?
\end{enumerate}

\hypertarget{mudando-o-valor-de-um-objeto}{%
\subsubsection{3.3 Mudando o valor de um
objeto}\label{mudando-o-valor-de-um-objeto}}

Outra coisa super fácil de fazer no R é \textbf{mudar o valor de uma
variável}. Para fazer isso, basta mudar o valor atribuído para essa
variável.

\begin{Shaded}
\begin{Highlighting}[]
\CommentTok{# Mudando o valor de 'a' para 33846}

\NormalTok{a <-}\StringTok{ }\DecValTok{50}

\NormalTok{a}
\end{Highlighting}
\end{Shaded}

\begin{verbatim}
## [1] 50
\end{verbatim}

\begin{Shaded}
\begin{Highlighting}[]
\NormalTok{a <-}\StringTok{ }\DecValTok{33846}

\NormalTok{a}
\end{Highlighting}
\end{Shaded}

\begin{verbatim}
## [1] 33846
\end{verbatim}

\begin{Shaded}
\begin{Highlighting}[]
\CommentTok{# Mudando o valor de 'a' para (5 + 12 + 8 + 1) * 4}

\NormalTok{a <-}\StringTok{ }\NormalTok{(}\DecValTok{5} \OperatorTok{+}\StringTok{ }\DecValTok{12} \OperatorTok{+}\StringTok{ }\DecValTok{8} \OperatorTok{+}\StringTok{ }\DecValTok{1}\NormalTok{) }\OperatorTok{*}\StringTok{ }\DecValTok{4}

\NormalTok{a}
\end{Highlighting}
\end{Shaded}

\begin{verbatim}
## [1] 104
\end{verbatim}

\begin{Shaded}
\begin{Highlighting}[]
\CommentTok{# Mudando o valor de 'a' para 'a / a'}

\NormalTok{a <-}\StringTok{ }\DecValTok{2} \OperatorTok{*}\StringTok{ }\NormalTok{a}

\NormalTok{a}
\end{Highlighting}
\end{Shaded}

\begin{verbatim}
## [1] 208
\end{verbatim}

\begin{Shaded}
\begin{Highlighting}[]
\CommentTok{# O R reconhece o valor antigo de 'a' e realiza a operação 2 * a.}

\CommentTok{# Mudando o valor de 'a' para 'a / a'}

\NormalTok{a <-}\StringTok{ }\NormalTok{a }\OperatorTok{/}\StringTok{ }\NormalTok{a}

\NormalTok{a}
\end{Highlighting}
\end{Shaded}

\begin{verbatim}
## [1] 1
\end{verbatim}

\begin{Shaded}
\begin{Highlighting}[]
\CommentTok{# O R reconhece o valor antigo de 'a' e realiza a operação a / a = 1.}
\end{Highlighting}
\end{Shaded}

Pronto! Mudar valores de objetos é algo relativamente simples de se
fazer com o R.

\textbf{Exercícios para você fazer}

\begin{enumerate}
\def\labelenumi{\arabic{enumi}.}
\item
  Crie um objeto com a soma de todos os números pares de 0 a 10.
\item
  Crie um objeto com a soma de todos os números ímpares de 0 a 10.
\item
  Multiplique esses objetos entre si, armazenando esse resultado em uma
  nova variável.
\end{enumerate}

\hypertarget{vetores-e-funuxe7uxf5es}{%
\subsection{4. Vetores e funções}\label{vetores-e-funuxe7uxf5es}}

\hypertarget{criando-vetores}{%
\subsubsection{4.1 Criando vetores}\label{criando-vetores}}

Vamos aprender a criar um vetor - uma variável que armazena um ou mais
valores.

Na verdade, a gente já criou alguns vetores antes. \emph{Basicamente
tudo no R é um vetor.}

Até agora, vimos vetores com apenas um valor, vamos aprender a criar
vetores com mais de 1 valor.

\textbf{PROBLEMA}

Imagine que um gerente de voleibol está interessado em descobrir a média
de salários dos jogadores do time de vôlei \emph{R Voleibol Clube}.

O time possui 8 jogadores, cada um com seu devido número.

\begin{itemize}
\tightlist
\item
  Os jogadores 1 ao 3 recebem R\$1500 por mês.
\item
  O jogador 4 recebe R\$3000.
\item
  O jogador 5 recebe R\$10000.
\item
  O jogador 6 recebe R\$1200.
\item
  Os jogadores 7 e 8 recebem R\$3300 cada.
\end{itemize}

Vamos resolver esse problema aprendendo sobre a função concatenar
\texttt{c()}.

Essa função permite juntar diferentes valores em um mesmo objeto. Assim,
podemos criar a variável (aqui em R, um outro nome para \emph{objeto})
chamada \texttt{salario}.

\begin{Shaded}
\begin{Highlighting}[]
\CommentTok{# Vamos adicionar os primeiros três elementos.}
\NormalTok{salario <-}\StringTok{ }\KeywordTok{c}\NormalTok{(}\DecValTok{1500}\NormalTok{, }\DecValTok{1500}\NormalTok{, }\DecValTok{1500}\NormalTok{)}

\NormalTok{salario}
\end{Highlighting}
\end{Shaded}

\begin{verbatim}
## [1] 1500 1500 1500
\end{verbatim}

Pronto, a função \texttt{c()} juntou os salários dos primeiros 3
jogadores. A separação entre um elemento e outro é feito utilizando uma
vírgula \texttt{,}.

Agora, vamos adicionar o salário do jogador 4.

\begin{Shaded}
\begin{Highlighting}[]
\CommentTok{# Vamos adicionar o quarto elemento.}
\NormalTok{salario <-}\StringTok{ }\KeywordTok{c}\NormalTok{(salario, }\DecValTok{3000}\NormalTok{)}

\NormalTok{salario}
\end{Highlighting}
\end{Shaded}

\begin{verbatim}
## [1] 1500 1500 1500 3000
\end{verbatim}

Olha que interessante! Concatenamos o valor antigo de \texttt{salario}
junto com o valor novo que queríamos.

Vamos finalizar isso.

\begin{Shaded}
\begin{Highlighting}[]
\CommentTok{# Vamos adicionar os últimos elementos.}
\NormalTok{salario <-}\StringTok{ }\KeywordTok{c}\NormalTok{(salario, }\DecValTok{10000}\NormalTok{, }\DecValTok{1200}\NormalTok{, }\DecValTok{3300}\NormalTok{, }\DecValTok{333300}\NormalTok{)}

\NormalTok{salario}
\end{Highlighting}
\end{Shaded}

\begin{verbatim}
## [1]   1500   1500   1500   3000  10000   1200   3300 333300
\end{verbatim}

Pronto, aí está a nossa variável \texttt{salario}!

Opa, tem um erro ali. O último elemento, o elemento de nº 8, está
errado!

Ao invés de colocar o valor de \texttt{3300}, colocamos o valor de
\texttt{333300}.

Vamos mudar o valor desse elemento.

Para \emph{acessar um elemento de um vetor}, basta realizar a função
\texttt{vetor{[}elemento{]}}.

Nesse caso, queremos acessar o elemento número \texttt{8} da variável
\texttt{salario}, para modificarmos o seu valor.

\begin{Shaded}
\begin{Highlighting}[]
\CommentTok{# Acessando um elemento}
\NormalTok{salario[}\DecValTok{8}\NormalTok{]  }\CommentTok{# Isso é como pedir ao R: "R, em 'salario', acesse o elemento '8'".}
\end{Highlighting}
\end{Shaded}

\begin{verbatim}
## [1] 333300
\end{verbatim}

\begin{Shaded}
\begin{Highlighting}[]
\CommentTok{# Para modificar o valor, basta atribuir um novo valor, como já vínhamos fazendo...}

\NormalTok{salario[}\DecValTok{8}\NormalTok{] <-}\StringTok{ }\DecValTok{3300}

\NormalTok{salario}
\end{Highlighting}
\end{Shaded}

\begin{verbatim}
## [1]  1500  1500  1500  3000 10000  1200  3300  3300
\end{verbatim}

Pronto! Agora \texttt{salario} está organizado assim como a gente
queria.

Nesse momento, tendo criado a variável \texttt{salario}, possuímos
alguns atributos que podemos descobrir sobre esse vetor.

\hypertarget{utilizando-funuxe7uxf5es}{%
\subsubsection{4.2 Utilizando funções}\label{utilizando-funuxe7uxf5es}}

Um deles é a média, a solução para o problema lá posto lá em cima. Para
saber a média, vamos utilizar uma \emph{função}.

\textbf{Funções} são fórmulas prontas que podemos utilizar. O R já vem
com algumas dessas fórmulas, como por exemplo a \textbf{raiz quadrada}
\texttt{sqrt()} (abreviação do inglês para ``square root'').

Para usar uma função, colocamos o seu nome e então o valor ou objeto que
queremos que essa função receba.

\begin{Shaded}
\begin{Highlighting}[]
\CommentTok{# Raiz quadrada de 25}

\KeywordTok{sqrt}\NormalTok{(}\DecValTok{25}\NormalTok{)}
\end{Highlighting}
\end{Shaded}

\begin{verbatim}
## [1] 5
\end{verbatim}

\begin{Shaded}
\begin{Highlighting}[]
\CommentTok{# Algo que se faz bastante é atribuir o resultado de funções a objetos }

\NormalTok{raiz <-}\StringTok{ }\KeywordTok{sqrt}\NormalTok{(}\DecValTok{25}\NormalTok{) }\CommentTok{# calcula a raiz quadrada de 25 e armazena em 'raiz'}

\NormalTok{raiz}
\end{Highlighting}
\end{Shaded}

\begin{verbatim}
## [1] 5
\end{verbatim}

\begin{Shaded}
\begin{Highlighting}[]
\CommentTok{# Assim, o valor da raiz de 25 estará facilmente disponível}
\end{Highlighting}
\end{Shaded}

Para descobrir o que uma função faz, basta colocar o seu nome no console
ao lado de um ponto de interrogação, dessa forma \texttt{?função}.

Outra forma é usar a função \texttt{help()}, dessa forma:
\texttt{help(função)}. Assim, caso quiséssemos entender o que
\texttt{sqrt()} faz, poderíamos tanto pedir \texttt{?sqrt} quanto
\texttt{help(sqrt)}. O resultado do que pedimos irá aparecer no
quadrante direito-baixo.

\textbf{VOLTANDO AO PROBLEMA}

Lembra que o gerente de voleibol estava interessado em descobrir a média
de salários dos jogadores do time de vôlei \emph{R Voleibol Clube}?

Acontece que a função \texttt{mean()} calcula a média dos elementos que
colocarmos nela.

\begin{Shaded}
\begin{Highlighting}[]
\CommentTok{# Calculando a média de dois números}

\KeywordTok{mean}\NormalTok{(}\KeywordTok{c}\NormalTok{(}\DecValTok{25}\NormalTok{, }\DecValTok{10}\NormalTok{))}
\end{Highlighting}
\end{Shaded}

\begin{verbatim}
## [1] 17.5
\end{verbatim}

\begin{Shaded}
\begin{Highlighting}[]
\CommentTok{# Note que os dois números foram concatenados com a função 'c()'.}

\CommentTok{# Calculando a média do time de voleibol e armazenando esse valor no objeto 'mediaSalario'.}

\NormalTok{mediaSalario <-}\StringTok{ }\KeywordTok{mean}\NormalTok{(salario)}

\NormalTok{mediaSalario}
\end{Highlighting}
\end{Shaded}

\begin{verbatim}
## [1] 3162.5
\end{verbatim}

Resolvemos nosso primeiro problema! A média de salário dos jogadores do
\emph{R Voleibol Clube} é de R\$3162,50 por mês.

\textbf{PROBLEMA}

Cada jogador recebeu um bônus diferente. Os jogadores 1 a 4 receberam
2000 de bônus, enquanto os jogadores de 5 a 8 receberam 4000 de bônus.
Calcule a média de salário nesse mês, considerando os bônus.

Agora que já sabemos o básico, podemos ser introduzidos à função
\texttt{rep()}, que repete determinado valor algumas vezes.

Vamos criar um objeto chamado \texttt{bonus} utilizando a função
\texttt{rep()} e \texttt{c()}:

\begin{Shaded}
\begin{Highlighting}[]
\CommentTok{# Criando bonus}

\NormalTok{bonus <-}\StringTok{ }\KeywordTok{c}\NormalTok{(}\KeywordTok{rep}\NormalTok{(}\DecValTok{2000}\NormalTok{, }\DecValTok{4}\NormalTok{), }\KeywordTok{rep}\NormalTok{(}\DecValTok{4000}\NormalTok{, }\DecValTok{4}\NormalTok{))}

\NormalTok{bonus}
\end{Highlighting}
\end{Shaded}

\begin{verbatim}
## [1] 2000 2000 2000 2000 4000 4000 4000 4000
\end{verbatim}

Bem, isso foi bem mais fácil do que criar \texttt{salario}. Mas como
isso aconteceu?

\begin{itemize}
\tightlist
\item
  Primeiro, colocamos \texttt{c()} para concatenar mais de um valor.
\item
  Após isso, utilizamos \texttt{rep()} para repetir o valor
  \texttt{2000} 4 vezes diferentes.
\item
  Separamos essa primeira parte utilizando uma vírgula \texttt{,}.
\item
  Nessa segunda parte, utilizamos \texttt{rep()} novamente para repetir
  o valor \texttt{4000} 4 vezes diferentes.
\item
  Fechamos o parênteses de \texttt{rep()} e também fechamos os
  parênteses de \texttt{c()}.
\end{itemize}

Assim, 8 elementos foram criados. Como adicionar esses 8 elementos aos
antigos 8 elementos?

\textbf{Basta adicionar os dois vetores!}

\begin{enumerate}
\def\labelenumi{\arabic{enumi}.}
\item
  O elemento \texttt{salario{[}1{]}} será adicionado ao elemento
  \texttt{bonus{[}1{]}}.
\item
  O elemento \texttt{salario{[}2{]}} será adicionado ao elemento
  \texttt{bonus{[}2{]}}.
\item
  Essa fórmula se repete até o elemento \texttt{salario{[}8{]}} ser
  adicionado ao elemento \texttt{bonus{[}8{]}}.
\end{enumerate}

\begin{Shaded}
\begin{Highlighting}[]
\CommentTok{# Para adicionar os dois vetores, basta:}

\NormalTok{salarioComBonus <-}\StringTok{ }\NormalTok{salario }\OperatorTok{+}\StringTok{ }\NormalTok{bonus}

\NormalTok{salarioComBonus}
\end{Highlighting}
\end{Shaded}

\begin{verbatim}
## [1]  3500  3500  3500  5000 14000  5200  7300  7300
\end{verbatim}

Podemos averiguar com a função \texttt{mean()} a nova média do valor
recebido pelos jogadores. Ainda, podemos utilizar a função
\texttt{sum()}, que irá somar todos os elementos que colocarmos nela.

\begin{Shaded}
\begin{Highlighting}[]
\CommentTok{# Averiguando nova média}

\NormalTok{mediaSalarioComBonus <-}\StringTok{ }\KeywordTok{mean}\NormalTok{(salarioComBonus)}

\NormalTok{mediaSalarioComBonus}
\end{Highlighting}
\end{Shaded}

\begin{verbatim}
## [1] 6162.5
\end{verbatim}

\begin{Shaded}
\begin{Highlighting}[]
\CommentTok{# Descobrindo o valor total que os jogadores receberam}

\NormalTok{somaSalario <-}\StringTok{ }\KeywordTok{sum}\NormalTok{(salarioComBonus)}

\NormalTok{somaSalario}
\end{Highlighting}
\end{Shaded}

\begin{verbatim}
## [1] 49300
\end{verbatim}

\begin{Shaded}
\begin{Highlighting}[]
\CommentTok{# Utilizando a função summary() - uma boa carta na manga}

\KeywordTok{summary}\NormalTok{(salarioComBonus) }\CommentTok{# summary() é usado para informações gerais de objetos com números}
\end{Highlighting}
\end{Shaded}

\begin{verbatim}
##    Min. 1st Qu.  Median    Mean 3rd Qu.    Max. 
##    3500    3500    5100    6162    7300   14000
\end{verbatim}

Descobrimos que a nova média é de R\$6162,50. Sabemos também que a soma
de todos os salários equivale a R\$49300.

Mais legal ainda, \texttt{summary()} nos informou várias informações.

\textbf{O que}\texttt{summary()}\textbf{mostrou?}

\begin{itemize}
\item
  Na primeira parte, em \texttt{Min.}( \emph{Mínimo} ): que o valor
  mínimo, na posição 0\%, foi R\$3500.
\item
  Na segunda parte, em \texttt{1st\ Qu.}( \emph{1º Quartil} ): que o
  valor na casa dos 25\% foi R\$3500.
\item
  Na terceira parte, em \texttt{Median}( \emph{Mediana} ): que o valor
  na casa dos 50\% (a mediana) foi R\$5100.
\item
  Na quarta parte, em \texttt{Mean}( \emph{Média} ): que a média de
  todos os valores foi R\$6162.
\item
  Na quinta parte, em \texttt{3rd\ Qu.}( \emph{3º Quartil} ): que o
  valor na casa dos 75\% foi R\$7300.
\item
  Na terceira parte, em \texttt{Max.}( \emph{Máximo} ): que o valor
  máximo, na posição 100\%, foi R\$14000.
\end{itemize}

Assim, \texttt{summary()} nos informa os valores na posição 0\%, 25\%,
50\% (mediana), 75\%, 100\% e ainda nos informa a média dos valores.

\hypertarget{instalando-e-lendo-pacotes}{%
\subsubsection{4.3 Instalando e lendo
pacotes}\label{instalando-e-lendo-pacotes}}

Aqui, vale a pena dar uma pausa para mencionar os \emph{pacotes do R}.
Pacotes são conjuntos de linhas de código, funções e cálculos que se
pode baixar diretamente no R. Assim, ao invés de escrever algoritmos
difíceis no R, podemos baixar pacotes que fazem isso para a gente.

\hypertarget{instalando-o-pacote-dplyr}{%
\subsubsection{4.3.1 Instalando o pacote
dplyr}\label{instalando-o-pacote-dplyr}}

Para instalar um pacote, basta executar o comando
\texttt{install.packages("pacote")}

\emph{É importante atentar para as aspas!}

Vamos agora instalar um pacote bastante usado para a manipulação de
dados - o pacote dplyr.

\begin{Shaded}
\begin{Highlighting}[]
\KeywordTok{install.packages}\NormalTok{(}\StringTok{"dplyr"}\NormalTok{) }\CommentTok{# basta colocar dplyr no lugar do pacote}
\end{Highlighting}
\end{Shaded}

\hypertarget{lendo-o-pacote-dplyr}{%
\subsubsection{4.3.2 Lendo o pacote dplyr}\label{lendo-o-pacote-dplyr}}

Assim que instalado, para utilizar as funções de um pacote, é necessário
trazê-lo ao R. Para isso, se utiliza a função \texttt{library(pacote)}.

\begin{Shaded}
\begin{Highlighting}[]
\KeywordTok{library}\NormalTok{(dplyr) }\CommentTok{# basta colocar dplyr no lugar do pacote}
\end{Highlighting}
\end{Shaded}

Agora, você tem as funções do \texttt{dplyr} para uso.

Caso você não tenha entendido esse processo, vamos tentar elaborar um
pouco mais. \textbf{É como se, para ler um livro, fosse necessário
pegá-lo de uma prateleira e colocar na mesa de trabalho. A prateleira é
seu computador, a sua mesa de trabalho é o R.}

Quando você já tem um livro em sua prateleira
(\texttt{install.packages("pacote")}), basta colocar na sua mesa de
trabalho com \texttt{library(pacote)}. Assim, até o final da sua sessão
no R, você pode utilizar as funções desse pacote. Após você sair de uma
sessão no R e iniciar outra sessão no R, será necessário realizar a
leitura do pacote que você desejar novamente com
\texttt{library(pacote)}.

\hypertarget{instalando-tidyverse}{%
\subsubsection{4.3.3 Instalando tidyverse}\label{instalando-tidyverse}}

O pacote \emph{tidyverse}(\url{https://www.tidyverse.org/}) é um pacote
bem famoso e utilizado pela comunidade do R. \textbf{O tidyverse é um
pacote que agrupa outros pacotes} muito importantes, como o \emph{dplyr}
para manipulação dos dados, \emph{ggplot2} para visualização de dados,
\emph{readr} para importação e leitura de dados e outros.

\textbf{Exercício para você fazer}

\begin{enumerate}
\def\labelenumi{\arabic{enumi}.}
\tightlist
\item
  Instale o pacote \emph{tidyverse} utilizando o que aprendeu nessa
  seção.
\end{enumerate}

\hypertarget{matrizes}{%
\subsection{5. Matrizes}\label{matrizes}}

As matrizes são elementos que armazenam um conjunto de \textbf{valores
numéricos}. Elas teriam um \emph{número i de linhas} e um \emph{número j
de colunas}. Assim, uma matriz A\textsubscript{ij} que tenha i = 2 e j =
3 é tida como \textbf{A\textsubscript{23}} - isso significa que essa
matriz terá 2 linhas (i = 2) e 3 colunas (j = 3).

Para criar uma matriz no R, se utiliza a função \texttt{matrix()}. Vamos
agora criar uma matriz A\textsubscript{23} que tenha os valores
\texttt{5}, \texttt{2} e \texttt{7} na primeira linha, e \texttt{10},
\texttt{1} e \texttt{9} na segunda linha.

\begin{Shaded}
\begin{Highlighting}[]
\CommentTok{# Para adicionar os dois vetores, basta:}

\NormalTok{novaMatriz <-}\StringTok{ }\KeywordTok{matrix}\NormalTok{(}\KeywordTok{c}\NormalTok{(}\DecValTok{5}\NormalTok{, }\DecValTok{2}\NormalTok{, }\DecValTok{7}\NormalTok{,}
                       \DecValTok{10}\NormalTok{, }\DecValTok{1}\NormalTok{, }\DecValTok{9}\NormalTok{), }\DataTypeTok{nrow =} \DecValTok{2}\NormalTok{, }\DataTypeTok{ncol =} \DecValTok{3}\NormalTok{)}

\NormalTok{novaMatriz}
\end{Highlighting}
\end{Shaded}

\begin{verbatim}
##      [,1] [,2] [,3]
## [1,]    5    7    1
## [2,]    2   10    9
\end{verbatim}

\begin{itemize}
\tightlist
\item
  Primeiro, criamos um objeto chamado \texttt{novaMatriz} que armazenará
  o valor resultante da função \texttt{matrix()}.
\item
  Na função matrix, concatenamos todas os elementos que queríamos com a
  função \texttt{c()}. Inserimos um \texttt{Enter} para visualizar a
  segunda linha de forma mais fácil (isso não altera os valores da
  matriz).
\item
  Por fim, adicionamos os argumentos \texttt{nrow\ =} e
  \texttt{ncol\ =}.

  \begin{itemize}
  \tightlist
  \item
    \texttt{nrow\ =} pede o número de linhas que queremos colocar nessa
    matriz. Já que i = 2, \texttt{nrow\ =\ 2}.
  \item
    \texttt{ncol\ =} pede o número de colunas que queremos colocar nessa
    matriz. Já que j = 3, \texttt{ncol\ =\ 3}.
  \end{itemize}
\end{itemize}

Agora que está criada, podemos acessar elementos específicos dessa
matriz. Para isso, utiliza-se o nome da matriz e, dentro de colchetes, o
valor da linha e da coluna que se deseja acessar:
\texttt{matriz{[}i,\ j{]}}

\begin{Shaded}
\begin{Highlighting}[]
\CommentTok{# Acessar o elemento da linha 1 e coluna 2}

\NormalTok{novaMatriz[}\DecValTok{1}\NormalTok{, }\DecValTok{2}\NormalTok{]}
\end{Highlighting}
\end{Shaded}

\begin{verbatim}
## [1] 7
\end{verbatim}

\begin{Shaded}
\begin{Highlighting}[]
\CommentTok{# Acessar o elemento da linha 2 e coluna 3}

\NormalTok{novaMatriz[}\DecValTok{2}\NormalTok{, }\DecValTok{3}\NormalTok{]}
\end{Highlighting}
\end{Shaded}

\begin{verbatim}
## [1] 9
\end{verbatim}

\begin{Shaded}
\begin{Highlighting}[]
\CommentTok{# Acessar todos os valores da segunda coluna}

\NormalTok{novaMatriz[, }\DecValTok{2}\NormalTok{]}
\end{Highlighting}
\end{Shaded}

\begin{verbatim}
## [1]  7 10
\end{verbatim}

\begin{Shaded}
\begin{Highlighting}[]
\CommentTok{# Acessar todos os valores da primeira linha}

\NormalTok{novaMatriz[}\DecValTok{1}\NormalTok{, ]}
\end{Highlighting}
\end{Shaded}

\begin{verbatim}
## [1] 5 7 1
\end{verbatim}

\begin{Shaded}
\begin{Highlighting}[]
\CommentTok{# Acessar os valores das linhas de 1 a 2 da coluna 1}

\NormalTok{novaMatriz[}\DecValTok{1}\OperatorTok{:}\DecValTok{2}\NormalTok{, }\DecValTok{1}\NormalTok{]}
\end{Highlighting}
\end{Shaded}

\begin{verbatim}
## [1] 5 2
\end{verbatim}

Já que podemos acessar, também podemos modificar esses valores.

\begin{Shaded}
\begin{Highlighting}[]
\CommentTok{# Mudando o elemento da linha 1 e coluna 2}

\NormalTok{novaMatriz}
\end{Highlighting}
\end{Shaded}

\begin{verbatim}
##      [,1] [,2] [,3]
## [1,]    5    7    1
## [2,]    2   10    9
\end{verbatim}

\begin{Shaded}
\begin{Highlighting}[]
\NormalTok{novaMatriz[}\DecValTok{1}\NormalTok{, }\DecValTok{2}\NormalTok{] <-}\StringTok{ }\DecValTok{3}

\NormalTok{novaMatriz}
\end{Highlighting}
\end{Shaded}

\begin{verbatim}
##      [,1] [,2] [,3]
## [1,]    5    3    1
## [2,]    2   10    9
\end{verbatim}

\begin{Shaded}
\begin{Highlighting}[]
\CommentTok{# Mudando todos os elementos da linha 2 para 0}

\NormalTok{novaMatriz}
\end{Highlighting}
\end{Shaded}

\begin{verbatim}
##      [,1] [,2] [,3]
## [1,]    5    3    1
## [2,]    2   10    9
\end{verbatim}

\begin{Shaded}
\begin{Highlighting}[]
\NormalTok{novaMatriz[}\DecValTok{2}\NormalTok{, ] <-}\StringTok{ }\DecValTok{0}

\NormalTok{novaMatriz}
\end{Highlighting}
\end{Shaded}

\begin{verbatim}
##      [,1] [,2] [,3]
## [1,]    5    3    1
## [2,]    0    0    0
\end{verbatim}

\hypertarget{objetos-string}{%
\subsection{\texorpdfstring{6. Objetos
\emph{string}}{6. Objetos string}}\label{objetos-string}}

Além de utilizar objetos para armazenar valores numéricos, podemos
também armazenar um conjunto de letras, símbolos e números - essas são
chamadas de \emph{strings}.

Por exemplo, vamos criar um objeto chamado \texttt{string} que contém o
texto: \texttt{Isso\ é\ uma\ string}.

\begin{Shaded}
\begin{Highlighting}[]
\CommentTok{# Para criar uma string, deve-se colocar o valor desejado em aspas}

\NormalTok{string <-}\StringTok{ "Isso é uma string"}

\NormalTok{string}
\end{Highlighting}
\end{Shaded}

\begin{verbatim}
## [1] "Isso é uma string"
\end{verbatim}

\begin{Shaded}
\begin{Highlighting}[]
\NormalTok{outraString <-}\StringTok{ "10"}

\NormalTok{outraString}
\end{Highlighting}
\end{Shaded}

\begin{verbatim}
## [1] "10"
\end{verbatim}

Criamos duas strings: uma que contém o valor
\texttt{Isso\ é\ uma\ string} e outro que contém o valor \texttt{"10"} -
esse ``10'' não é um número pois está entre aspas.

Podemos adicionar novos elementos assim como fizemos com vetores de
números.

\begin{Shaded}
\begin{Highlighting}[]
\CommentTok{# Novo elemento "Outros valores"}

\NormalTok{string <-}\StringTok{ }\KeywordTok{c}\NormalTok{(string, }\StringTok{"Outros valores"}\NormalTok{)}

\NormalTok{string}
\end{Highlighting}
\end{Shaded}

\begin{verbatim}
## [1] "Isso é uma string" "Outros valores"
\end{verbatim}

\begin{Shaded}
\begin{Highlighting}[]
\CommentTok{# Assim como antes, podemos alterar esse elemento ou até adicionar outro}

\NormalTok{string[}\DecValTok{2}\NormalTok{] <-}\StringTok{ "Valor mais interessante"}

\NormalTok{string}
\end{Highlighting}
\end{Shaded}

\begin{verbatim}
## [1] "Isso é uma string"       "Valor mais interessante"
\end{verbatim}

\hypertarget{listas}{%
\subsection{7. Listas}\label{listas}}

Listas são um conjunto de objetos - por isso é o último assunto que
veremos. Ela possui a capacidade de armazenar várias informações
diferentes.

Até agora, esses foram os principais objetos criados: * \texttt{objeto}:
um objeto que contém o número \texttt{10}.

\begin{itemize}
\item
  \texttt{salario}: um vetor contendo 8 salários do time \emph{R
  Voleibol Clube}.
\item
  \texttt{novaMatriz}: uma matriz A\textsubscript{23} com números
  aleatórios.
\item
  \texttt{string}: um vetor de strings contendo 2 elementos.
\end{itemize}

Uma lista é um objeto que pode abranger todas essas informações - ele
pode ser criado com a função \texttt{list()}.

\begin{Shaded}
\begin{Highlighting}[]
\CommentTok{# Essa lista vai se chamar listaPoderosa}

\NormalTok{listaPoderosa <-}\StringTok{ }\KeywordTok{list}\NormalTok{(objeto, salario, novaMatriz, string)}

\NormalTok{listaPoderosa}
\end{Highlighting}
\end{Shaded}

\begin{verbatim}
## [[1]]
## [1] 10
## 
## [[2]]
## [1]  1500  1500  1500  3000 10000  1200  3300  3300
## 
## [[3]]
##      [,1] [,2] [,3]
## [1,]    5    3    1
## [2,]    0    0    0
## 
## [[4]]
## [1] "Isso é uma string"       "Valor mais interessante"
\end{verbatim}

\textbf{Já que não nomeamos os objetos que iam entrar na lista}, o
seguinte aconteceu:

\begin{itemize}
\item
  O elemento \texttt{listaPoderosa{[}{[}1{]}{]}} armazenou o valor de
  \texttt{objeto}.
\item
  O elemento \texttt{listaPoderosa{[}{[}2{]}{]}} armazenou o valor de
  \texttt{salario}.
\item
  O elemento \texttt{listaPoderosa{[}{[}3{]}{]}} armazenou o valor de
  \texttt{novaMatriz}.
\item
  O elemento \texttt{listaPoderosa{[}{[}4{]}{]}} armazenou o valor de
  \texttt{string}.
\end{itemize}

\begin{Shaded}
\begin{Highlighting}[]
\CommentTok{# Agora vamos nomear cada objeto com o seu próprio nome}

\NormalTok{listaPoderosa <-}\StringTok{ }\KeywordTok{list}\NormalTok{(}\DataTypeTok{objeto =}\NormalTok{ objeto, }
                      \DataTypeTok{salario =}\NormalTok{ salario, }
                      \DataTypeTok{novaMatriz =}\NormalTok{ novaMatriz, }
                      \DataTypeTok{string =}\NormalTok{ string)}

\NormalTok{listaPoderosa}
\end{Highlighting}
\end{Shaded}

\begin{verbatim}
## $objeto
## [1] 10
## 
## $salario
## [1]  1500  1500  1500  3000 10000  1200  3300  3300
## 
## $novaMatriz
##      [,1] [,2] [,3]
## [1,]    5    3    1
## [2,]    0    0    0
## 
## $string
## [1] "Isso é uma string"       "Valor mais interessante"
\end{verbatim}

Pronto! Para acessar os objetos de uma lista utilizamos o cifrão
\texttt{\$} (pode-se fazer o mesmo com os objetos de um
\texttt{data.frame}, como veremos adiante).

Com isso, podemos:

\begin{enumerate}
\def\labelenumi{\arabic{enumi}.}
\item
  Observar um objeto específico de uma lista.
\item
  Modificar um elemento específico ou acrescentar valores diretamente em
  uma lista.
\item
  Criar um objeto extraindo apenas um objeto de uma lista.
\item
  Excluir um objeto de uma lista.
\item
  Excluir um elemento de uma lista.
\end{enumerate}

\begin{Shaded}
\begin{Highlighting}[]
\CommentTok{# 1. Observar um objeto específico de uma lista}

\NormalTok{listaPoderosa[[}\DecValTok{2}\NormalTok{]] }\CommentTok{# Vê o objeto 2 (salario)}
\end{Highlighting}
\end{Shaded}

\begin{verbatim}
## [1]  1500  1500  1500  3000 10000  1200  3300  3300
\end{verbatim}

\begin{Shaded}
\begin{Highlighting}[]
\NormalTok{listaPoderosa}\OperatorTok{$}\NormalTok{salario }\CommentTok{# Vê o objeto 2 (salario)}
\end{Highlighting}
\end{Shaded}

\begin{verbatim}
## [1]  1500  1500  1500  3000 10000  1200  3300  3300
\end{verbatim}

\begin{Shaded}
\begin{Highlighting}[]
\CommentTok{# 2. Modificar um elemento específico ou acrescentar valores diretamente em uma lista.}

\NormalTok{listaPoderosa[[}\DecValTok{2}\NormalTok{]][}\DecValTok{1}\NormalTok{] <-}\StringTok{ }\DecValTok{5000} \CommentTok{# Acessa o objeto 2 (salario), elemento 1, e modifica ele para 5000}
\NormalTok{listaPoderosa}\OperatorTok{$}\NormalTok{salario[}\DecValTok{1}\NormalTok{] <-}\StringTok{ }\DecValTok{5000} \CommentTok{# Acessa o objeto 2 (salario), elemento 1, e modifica ele para 5000}

\CommentTok{# 3. Criar um objeto extraindo apenas um objeto de uma lista.}

\NormalTok{queroEssaMatriz <-}\StringTok{ }\NormalTok{listaPoderosa}\OperatorTok{$}\NormalTok{novaMatriz }\CommentTok{# pega o objeto matriz e armazena em um objeto separado}
\NormalTok{queroEssaMatriz}
\end{Highlighting}
\end{Shaded}

\begin{verbatim}
##      [,1] [,2] [,3]
## [1,]    5    3    1
## [2,]    0    0    0
\end{verbatim}

\begin{Shaded}
\begin{Highlighting}[]
\CommentTok{# 4. Excluir um objeto de uma lista.}

\NormalTok{listaPoderosa <-}\StringTok{ }\NormalTok{listaPoderosa[}\OperatorTok{-}\NormalTok{salario] }\CommentTok{# recria o objeto listaPoderosa, dessa vez sem o objeto "salario"}
\NormalTok{listaPoderosa}
\end{Highlighting}
\end{Shaded}

\begin{verbatim}
## $objeto
## [1] 10
## 
## $salario
## [1]  5000  1500  1500  3000 10000  1200  3300  3300
## 
## $novaMatriz
##      [,1] [,2] [,3]
## [1,]    5    3    1
## [2,]    0    0    0
## 
## $string
## [1] "Isso é uma string"       "Valor mais interessante"
\end{verbatim}

\begin{Shaded}
\begin{Highlighting}[]
\CommentTok{# 5. Excluir um elemento de uma lista}

\NormalTok{listaPoderosa}\OperatorTok{$}\NormalTok{string <-}\StringTok{ }\NormalTok{listaPoderosa}\OperatorTok{$}\NormalTok{string[}\OperatorTok{-}\DecValTok{2}\NormalTok{] }\CommentTok{# o objeto string possui o valor de string SEM o elemento 2}
\NormalTok{listaPoderosa}\OperatorTok{$}\NormalTok{string }\CommentTok{# acessando apenas o objeto string}
\end{Highlighting}
\end{Shaded}

\begin{verbatim}
## [1] "Isso é uma string"
\end{verbatim}

Muitas dessas mesmas operações com listas podem também ser feitas em
objetos do tipo \texttt{data.frame}.

Por enquanto é isso! :)

\end{document}
